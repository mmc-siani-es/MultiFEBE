\documentclass{article}
\usepackage{markdown}
\usepackage[utf8]{inputenc}
\usepackage{geometry}
\geometry{verbose,tmargin=2.5cm,bmargin=2.5cm,lmargin=3cm,rmargin=3cm}
\usepackage{amsmath,amssymb,amsthm}
\usepackage{graphicx}
\graphicspath{{graphics/}}
\usepackage{fancyvrb}
\usepackage{hyperref}
\usepackage{lscape}
\usepackage{adjustbox}
\usepackage{verbatim}


\title{MultiFEBE \\ Examples}
\author{}
\date{}

\begin{document}

\maketitle

This document lists all the included examples for users, and it also describes the guidelines for developing and including new examples. 

Examples are organized by the type of problem \texttt{PP}, type of analysis \texttt{AA}, modeling \texttt{MM}, and number \texttt{NNN}, i.e. a generic example is encoded as \texttt{PP-AA-MM-NNN}. The examples whose title starts with ``[TUTORIAL]'' include a document with a description of the modeling approach, input files, output files and a comparison of the obtained results against other reference results (analytical or otherwise), which serves as validation.

\section{List of examples for users}

\begin{itemize}
    \item \texttt{ME} Linear elastic mechanics
    \begin{itemize}
        \item \texttt{ST} Static
        \begin{itemize}
            \item \texttt{SR} Structures
            \begin{itemize}
                \item \texttt{001} \href{ME-ST-SR-001/}{[TUTORIAL] Cantilever beam (FEM model)}.
            \end{itemize}
            \item[\texttt{EL}] Elastostatics (continuum)
            \begin{itemize}
                \item \texttt{001} \href{ME-ST-EL-001/}{[TUTORIAL] Plain strain square (BEM model)}.
                \item \texttt{002} \href{ME-ST-EL-002/}{[TUTORIAL] Cube (BEM model)}.
            \end{itemize}            
            \item \texttt{CO} Coupled 
            \begin{itemize}
                \item \texttt{001} \href{ME-ST-CO-001/}{[TUTORIAL] Floating pile response under head force or moment (BEM-FEM model)}.
            \end{itemize}             
        \end{itemize}
        \item \texttt{TH} Time Harmonic
        \begin{itemize}
            \item \texttt{SR} Structures
            \begin{itemize}
                \item \texttt{001} \href{ME-TH-SR-001/}{[TUTORIAL] Cantilever beam (FEM model)}.
            \end{itemize}             
            \item \texttt{EL} Elastodynamics (continuum)
            \begin{itemize}
                \item \texttt{001} \href{ME-TH-EL-001/}{[TUTORIAL] Cube (BEM model)}.
                \item \texttt{002} \href{ME-TH-EL-002/}{[TUTORIAL] Cantilever wall (BEM model)}.
                \item \texttt{003} \href{ME-TH-EL-003/}{[TUTORIAL] Soria arch dam: fixed-base (BEM model).}
            \end{itemize}            
            \item \texttt{PO} Poroelastodynamics (continuum)
            \begin{itemize}
                \item \texttt{none}
            \end{itemize}                 
            \item \texttt{AC} Acoustics (continuum)
            \begin{itemize}
                \item \texttt{none}
            \end{itemize}             
            \item \texttt{CO} Coupled 
            \begin{itemize}
                \item \texttt{001} \href{ME-TH-CO-001/}{[TUTORIAL] Impedances of inclined pile foundations (BEM-FEM model)}.
                \item \texttt{002} \href{ME-TH-CO-002/}{[TUTORIAL] Seismic response of a single inclined pile (BEM-FEM model)}.
                \item \texttt{003} \href{ME-TH-CO-003/}{[TUTORIAL] Impedances of a suction caisson / bucket foundation (BEM-FEM model)}.
                \item \texttt{004} \href{ME-TH-CO-004/}{[TUTORIAL] Seismic response of an Offshore Wind Turbine (BEM-FEM model)}.
                \item \texttt{005} \href{ME-TH-CO-005/}{[TUTORIAL] Soria arch dam: compliant base (BEM model)}.
                \item \texttt{006} \href{ME-TH-CO-006/}{[TUTORIAL] Soria arch dam: compliant base with 112 m water depth (BEM model)}.
            \end{itemize}                  
        \end{itemize}        
    \end{itemize}    
\end{itemize}

\section{Notes for developers}

Examples should be categorized within the following tree:
\begin{itemize}
    \item \texttt{ME} Linear elastic mechanics
    \begin{itemize}
        \item \texttt{ST} Static
        \begin{itemize}
            \item \texttt{SR} Structures. Examples using only structural and discrete finite elements.
            \item \texttt{EL} Elastostatics (continuum). Examples using either boundary elements or finite element, where the domain is discretized without structural simplifications, i.e. a continuum or solid model.
            \item \texttt{CO} Coupled. Examples where structures are coupled with the surrounding media.
        \end{itemize}
        \item \texttt{TH} Time Harmonic
        \begin{itemize}
            \item \texttt{SR} Structures. Examples using only structural and discrete finite elements.
            \item \texttt{EL} Elastodynamics (continuum). Examples using either boundary elements or finite element, where the domain is discretized without structural simplifications, i.e. a continuum or solid model. Only elastic solids.
            \item \texttt{PO} Poroelastodynamics (continuum). The same as above, but when using poroelastic media.
            \item \texttt{AC} Acoustics (continuum). The same as above, but when using inviscid fluids.
            \item \texttt{CO} Coupled. Examples where structures are coupled with the surrounding media, and/or where different material models are interacting.
        \end{itemize}        
    \end{itemize}
    \item \texttt{LA} Laplace (to be developed)
\end{itemize}

Within the \texttt{examples} folder, the example must have a dedicated folder with the name \texttt{PP-AA-MM-NNN} according to its categorization and numbering. The structure within the example folder must be:
\begin{itemize}
    \item \texttt{PP-AA-MM-NNN.pdf} - It is a mandatory document describing example. It can be as simple as a readme type of document, where a brief description of the example is given, or it can be a fully documented example in a tutorial format.
    \item \texttt{doc\_src} - It is a mandatory folder containing the sources for creating \texttt{PP-AA-MM-NNN.pdf}. It can be a plain text file \texttt{PP-AA-MM-NNN.txt}, a Microsoft Word file \texttt{PP-AA-MM-NNN.doc} or \texttt{PP-AA-MM-NNN.docx}, a Open Document Text file \texttt{PP-AA-MM-NNN.odt}, or a LaTeX file \texttt{PP-AA-MM-NNN.tex}. The preferred document source is LaTeX.
    \item \texttt{case\_files} - It is a mandatory folder containing all input files, including MultiFEBE, mesh and other files required for running the example.
\end{itemize}

\end{document}
