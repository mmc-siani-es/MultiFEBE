%===============================================================================
% Configuracion de pagina
\usepackage[a4paper,
twoside,
hmargin={2cm,2cm},
vmargin={2cm,2cm},
bindingoffset=1cm]{geometry}
%===============================================================================
% Idioma
%\usepackage[spanish]{babel}
%\decimalpoint
%===============================================================================
% Paquetes para fuentes
\usepackage{amsfonts}
\usepackage{amsmath}
\usepackage{amssymb}
\usepackage{mathtools}
\usepackage{amsthm}
\usepackage{bm}
\usepackage{wasysym}
\usepackage{eurosym}
\usepackage{textcomp}
\usepackage{url}      % Poner URL con enlace en el pdf y formatear
%\usepackage{etex}     % Para memoria dinamica - listings
%\reserveinserts{28}   % Para memoria dinamica - listings
%\usepackage{listings} % Codigo fuente y algoritmos
% Tipo de letra sans serif
% Ver web: http://dtrx.de/od/tex/sfmath.html
% Descomentar esta linea
%\usepackage{cmbright}
% O bien descomentar estas dos lineas
%\renewcommand{\familydefault}{\sfdefault} % letra sans serif
%\usepackage{sfmath} % math en sans serif
% Tipo de letra parecida a Elsevier
%\usepackage{fourier}
% Letra times
%\usepackage{mathptmx}
% %===============================================================================
% % Imagenes, figuras y tablas
\usepackage[table,x11names]{xcolor}
\usepackage{booktabs}        % Reglas para tablas
\usepackage{longtable}       % Tablas en varias paginas
\setcounter{LTchunksize}{30} % Ajustar chunks para el paquete longtable
\usepackage{ltabptch}        % Bug longtable
\usepackage{rotating}        % Girar figuras y tablas sideways
\usepackage{lscape}          % Paquete para girar tablas y ocupar varias paginas
\usepackage{colortbl}        % Coloreado de tablas
\usepackage{multirow}        % Tratamiento celdas tablas
\usepackage{caption}
\usepackage{graphicx}
\usepackage{epsfig}
\usepackage{subcaption}
\usepackage{float}
\usepackage{subfloat}
\usepackage{multicol}
%\usepackage{subfig}          % Varias figuras en un flotante entre paginas
%\usepackage{pst-all}         % pstricks
%\usepackage{pstricks,framed} % pstricks dibujo frames
%\usepackage{tikz}            % Paquete grafico muy potente
% %===============================================================================
% % Decoracion del documento
\usepackage[explicit]{titlesec}              % Variables decorar cada chapter
\usepackage{fancyhdr}                        % Encabezados y pie de pagina.
\usepackage{fancybox}                        % Cajas alrededor de textos
\usepackage{fancyvrb}                        % Verbatim specials
\usepackage{enumerate}                       % Tipos de numeracion
\usepackage[footnote,printonlyused]{acronym} % Acronimos
%===============================================================================
% Indice de palabras
\usepackage{makeidx}
\makeindex
%===============================================================================
\usepackage{hyperref} % si se utiliza, comentar subfig
\usepackage{cleveref}
